\mysection{Concept}
First, this chapter describes the selection of the appropriate framework. Futhermore, the structure of the model which was used for classification is explained based on its architecture. The chapter closes with the class diagram of the mobile application.

	\mysubsection{Frameworks}
As a deep learning framework Tensorflow was used to retrain the model. This decision was mainly based on recommendations. Even companies like e.g. NVIDIA Corporation, Intel Corporation etc. use Tensorflow. It is one of the common frameworks for deep learning applications and also provides solutions for integration in mobile apps. Furthermore, Tensorflow provides a variety of tutorials for working with neural networks. Beside those advantages, there is a large community about Tensorflow talking about issues and solutions. \\

To run the tensor within a mobile application, the first approach was to use Tensorflow Lite which is still in development state. But many attempts resulted in corrupt models which caused the app to terminate. Because of this experiences Tensorflow Mobile was used to run the model within the mobile application. 
		
	\mysubsection{Model based Architectures} Andi
	 - general architectures of models -> n Mobilenet, Inception
	 
	\mysubsection{Application based Architecture} 
This chapter describes the functions to a basic understanding of how the application works based on the application's architecture. It focuses on the important classes and methods of the mobile application. \\
Tensorflow provides a mobile application for demo purposes. Because of the complexity the application was adapted to our needs. The approach was to understand how the application was implemented regarding the functionality of the camera, image input for the network and background classification besides live camera stream. \\

The ClassifierActivity is set as the launcher activity of the application. It extends the class CameraActivity and loads the CameraFragment which controls the camera view. The CustomTextureView enables the possibility to capture frames from a camera stream and process it. Because of this function, the CustomTextureView is part of the class CameraFragment. If an image is captured, the method onImageAvailable of the class ClassifierActivity is invoked. The classification itself is done by the TensorflowImageClassifier which implements the interface Classifier from the Tensorflow API. The results are displayed by the class RecognitionScoreView which implements the interface ResultsView.