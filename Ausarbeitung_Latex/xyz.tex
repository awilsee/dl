\documentclass{scrartcl}
\usepackage{includes/mypack}
\usepackage{animate}
\usepackage{courier}
\usepackage{listings}
\usepackage[utf8]{inputenc}
\definecolor{mygreen}{rgb}{0,0.6,0}
\definecolor{mygray}{rgb}{0.5,0.5,0.5}
\definecolor{mymauve}{rgb}{0.58,0,0.82}
\lstset{ %
  backgroundcolor=\color{white},   % choose the background color; you must add \usepackage{color} or \usepackage{xcolor}; should come as last argument
  basicstyle=\footnotesize\ttfamily,        % the size of the fonts that are used for the code
  breakatwhitespace=false,         % sets if automatic breaks should only happen at whitespace
  breaklines=true,                 % sets automatic line breaking
  captionpos=b,                    % sets the caption-position to bottom
  commentstyle=\color{mygreen},    % comment style
  %deletekeywords={...},            % if you want to delete keywords from the given language
  %escapeinside={\%*}{*)},          % if you want to add LaTeX within your code
  extendedchars=true,              % lets you use non-ASCII characters; for 8-bits encodings only, does not work with UTF-8
  %frame=single,	                   % adds a frame around the code
  keepspaces=true,                 % keeps spaces in text, useful for keeping indentation of code (possibly needs columns=flexible)
  keywordstyle=\color{blue},       % keyword style
  language=python,                 % the language of the code
  %morekeywords={*,...},            % if you want to add more keywords to the set
  numbers=left,                    % where to put the line-numbers; possible values are (none, left, right)
  numbersep=5pt,                   % how far the line-numbers are from the code
  numberstyle=\tiny\color{mygray}, % the style that is used for the line-numbers
  rulecolor=\color{black},         % if not set, the frame-color may be changed on line-breaks within not-black text (e.g. comments (green here))
  showspaces=false,                % show spaces everywhere adding particular underscores; it overrides 'showstringspaces'
  showstringspaces=false,          % underline spaces within strings only
  showtabs=false,                  % show tabs within strings adding particular underscores
  stepnumber=1,                    % the step between two line-numbers. If it's 1, each line will be numbered
  stringstyle=\color{mymauve},     % string literal style
  tabsize=4,	                   % sets default tabsize to 2 spaces
  %title=\lstname                   % show the filename of files included with \lstinputlisting; also try caption instead of title
}


\Autoren{Alice Bollenmiller, Andreas Wilhelm}
\VersuchLang{Realizating an native Android App using Tensorflow} %z.B. Ein Wichtiger Versuch
\VersuchKurz{Deep Learning} %z.B. EWF
\Fach{Deep Learning -  Dog Breed Categorization} %z.B. Physik
\Studiengruppe{IG}
\Semester{WS 17/18}
\Betreuer{Lorenzo Servadei}
\Datum{\today} %falls gewünscht auf Versuchsdatum ändern, z.B. 21. November 2012

\PDFStartpage{1} %Seite die im Reader beim Start geöffnet wird
\MyParLineSkip{0.5} %Höhe der Absätze die durch \mypar erzeugt werden: 0...1

\praeinit %Initialisierung der Styles Teil 1

%umschalter zwischen WORKMODE und FINALMODE
%im Workmode gibts kein titelblatt und kein inhaltsverzeichnis,
%sodass man sich auf die arbeit an sich konzentrieren kann,
%und das setzen schneller geht. 
%wichtig: immer mindestens dreimal setzen lassen,
%wenn auf final umgeschaltet wird, 
%damit Veweise und Inhaltsverzeichnis stimmen!!!!
\setboolean{finalmode}{true}

%Umschalter zwischen draft und normal mode
%bewirkt, dass falls eingeschaltet, alle grafiken durch rahmen
%der entsprechenden größe ersetzt und dargestellt werden.
%erhöht die performance beim setzen deutlich und verhindert ablenkung 
%beim arbeiten durch bilder.
\setboolean{draftgraphics}{false}

\begin{document}
\postinit %Initialisierung der Styles Teil 2
%%AB HIER GEHT DIE ARBEIT LOS!

\section{Introduction}
huhu

\begin{equation}
	\triangle u = f
\end{equation}

Huhu




%%HIER ENDET DIE ARBEIT! DER REST IST KOMMENTAR MIT EIN PAAR PRAKTISCHEN FORMATIERUNGEN!!
\end{document}



%%%%%EINIGE SEHR PRAKTISCHE FORMATIERUNGSBEFEHLE UM ALLES SCHÖN UND EINHEITLICH AUSSEHEN ZU LASSEN!!%%%%%%%
%%%%%%%%%%%%%%%%%%%%%%%%%%%%%%%%%%%%%%%%%%%%%%%%%%%%%%%%%%%%%%%%%%%%%%%%%%%%%%%%%%%%%%%%%%%%%%%%%%%%%%%%%%%
%%%%%%%%%%%%%%%%%%%%%COMMENT%%%%%%%%%%%%%%%%%%%%%%COMMENT%%%%%%%%%%%%%%%%%%%%%%COMMENT%%%%%%%%%%%%%%%%%%%%%
%%%%%%%%%%%%%%%%%%%%%%%%%%%%%%%%%%%%%%%%%%%%%%%%%%%%%%%%%%%%%%%%%%%%%%%%%%%%%%%%%%%%%%%%%%%%%%%%%%%%%%%%%%%
\begin{comment}

% FORMELN UND PARAMETERBESCHREIBUNGEN
\begin{align}
\Delta \varphi = 360^\circ \cdot \frac{\Delta t}{T} = 360^\circ \cdot \Delta t \cdot f
\end{align}
wobei:
\begin{conditions} % *sternchen für zeilenübergreifende beschreibungstexte!!!
\Delta \varphi	& Phasenverschiebung \\
\Delta t		& Zeitdifferenz zwischen Eingangsspannung und gedämpfter Ausgangsspannung\\
T				& Periodendauer der Eingangsspannung \\
f				& Frequenz der Eingangsspannung
\end{conditions}

% Aufrechte griechische Buchstaben für Einheiten
\upmu
\upalpha
...etc

% BILDER EINFÜGEN
\begin{figure}[h] %t=top b=bottom h=here p =eigene page
\centering
\includegraphics[width=15cm]{media/hohlleiter}
\caption{Versuchsaufbau Hohlleiter}
\label{fig:label}
\end{figure}

% MEHRERE BILDER NEBENEINANDER
\begin{figure}[h] %t=top b=bottom h=here p =eigene page 
\flushright
\subfloat[Dispersionsrelation]{\includegraphics[width=7cm]{media/dispers}}
\subfloat[Phasen- und Gruppengeschwindigkeit]{\includegraphics[width=7cm]{media/phasgrupp}}
\end{figure}

%BILDER VOM TEXT UMFLOSSEN
\begin{wrapfigure}[10]{r}{8cm}
\centering
\includegraphics[width=7cm]{media/bragg}
\caption{\textsl{Aufbau eines Bragg-Spektrometers}}
\label{fig:bragg}
\end{wrapfigure}

%TABELLEN
\begin{table}[h]
\centering
\begin{tabular}{|c|c|}
\hline
$z$ [mm] & $\Delta z$ [mm]\\
\hline
3,3 & 0 \\
4,9 & 1,6 \\
6,5 & 1,6 \\
8,0 & 1,5 \\
\hline
\end{tabular}
\caption{Position der Minima zueinander und jeweiliger gemessener Abstand}
\label{tab:label}
\end{table}

% VERBUNDENE ZELLEN IN TABELLEN
\begin{table}[h]
\centering
\begin{tabular}{|c|c|c|c|p{1cm}p{1cm}p{1cm}p{1cm}p{1cm}p{1cm}p{1cm}|}
\hline
A & B & C & D & \multicolumn{7}{|c|}{F}  \\ \hline
\multirow{ 2}{*}{1} & 0 & 6 & 230 & 35 & 40 & 55 & 25 & 40 & 35 & \\
& 1 & 5 & 195 & 25 & 50 & 35 & 40 & 45 &  &  \\ \hline
\end{tabular}
\caption{A test caption}
\label{tab:table2}
\end{table}

% MEHRERE TABELLEN NEBENEINANDER
\begin{table}[h]
\centering

\subfloat[PVC, $d=15,0\,$mm]{
\begin{tabular}[b]{|c|c|c|}
\hline
mit Platte & ohne Platte & $\Delta s$\\
\hline
3,6 & 4,85 & 1,25\\
3,55 & 4,8 & 1,25\\
3,5 & 4,75 & 1,25\\
3,55 & 4,85 & 1,3\\
\hline
\end{tabular}
}

\subfloat[PE, $d=12,2\,$mm]{
\begin{tabular}[b]{|c|c|c|}
\hline
mit Platte & ohne Platte & $\Delta s$\\
\hline
3,5 & 4,2 & 0,7\\
3,5 & 4,25 & 0,75\\
3,5 & 4,2 & 0,7\\
\hline
\end{tabular}
}

\caption{Materialproben und zugehörige Messwerte in [cm]}
\end{table}

% FLOATS ERZWINGEN = TABELLEN UND BILDER ZWINGEND EINFÜGEN
\FloatBarrier

% ABSÄTZE MIT EINSTELLBARER UND NACHTRÄGLICH GLOBAL ÄNDERBARER DISTANZ
\mypar

% PDFs ANHÄNGEN
\includepdf[pages=-]{media/protokoll}

%BILDER UND TABELLEN REFERENZIEREN
\ref{labelname}
\pageref{labelname}

%REFERENZIERUNGSREGELN ZUR ÜBERSICHTLICHKEIT
bilder: fig:label
tabellen: tab:label
gleichungen: eq:label

\end{comment}
%%%%%%%%%%%%%%%%%%%%%%%%%%%%%%%%%%%%%%%%%%%%%%%%%%%%%%%%%%%%%%%%%%%%%%%%%%%%%%%%%%%%%%%%%%%%%%%%%%%%%%%%%%%
%%%%%%%%%%%%%%%%%%%%%COMMENT%%%%%%%%%%%%%%%%%%%%%%COMMENT%%%%%%%%%%%%%%%%%%%%%%COMMENT%%%%%%%%%%%%%%%%%%%%%
%%%%%%%%%%%%%%%%%%%%%%%%%%%%%%%%%%%%%%%%%%%%%%%%%%%%%%%%%%%%%%%%%%%%%%%%%%%%%%%%%%%%%%%%%%%%%%%%%%%%%%%%%%%
