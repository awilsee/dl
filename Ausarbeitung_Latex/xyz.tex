\documentclass{scrartcl}
\usepackage{includes/mypack}
\usepackage{animate}
\usepackage{courier}
\usepackage{listings}
\usepackage[utf8]{inputenc}
\definecolor{mygreen}{rgb}{0,0.6,0}
\definecolor{mygray}{rgb}{0.5,0.5,0.5}
\definecolor{mymauve}{rgb}{0.58,0,0.82}
\lstset{ %
  backgroundcolor=\color{white},   % choose the background color; you must add \usepackage{color} or \usepackage{xcolor}; should come as last argument
  basicstyle=\footnotesize\ttfamily,        % the size of the fonts that are used for the code
  breakatwhitespace=false,         % sets if automatic breaks should only happen at whitespace
  breaklines=true,                 % sets automatic line breaking
  captionpos=b,                    % sets the caption-position to bottom
  commentstyle=\color{mygreen},    % comment style
  %deletekeywords={...},            % if you want to delete keywords from the given language
  %escapeinside={\%*}{*)},          % if you want to add LaTeX within your code
  extendedchars=true,              % lets you use non-ASCII characters; for 8-bits encodings only, does not work with UTF-8
  %frame=single,	                   % adds a frame around the code
  keepspaces=true,                 % keeps spaces in text, useful for keeping indentation of code (possibly needs columns=flexible)
  keywordstyle=\color{blue},       % keyword style
  language=python,                 % the language of the code
  %morekeywords={*,...},            % if you want to add more keywords to the set
  numbers=left,                    % where to put the line-numbers; possible values are (none, left, right)
  numbersep=5pt,                   % how far the line-numbers are from the code
  numberstyle=\tiny\color{mygray}, % the style that is used for the line-numbers
  rulecolor=\color{black},         % if not set, the frame-color may be changed on line-breaks within not-black text (e.g. comments (green here))
  showspaces=false,                % show spaces everywhere adding particular underscores; it overrides 'showstringspaces'
  showstringspaces=false,          % underline spaces within strings only
  showtabs=false,                  % show tabs within strings adding particular underscores
  stepnumber=1,                    % the step between two line-numbers. If it's 1, each line will be numbered
  stringstyle=\color{mymauve},     % string literal style
  tabsize=4,	                   % sets default tabsize to 2 spaces
  %title=\lstname                   % show the filename of files included with \lstinputlisting; also try caption instead of title
}


\Autoren{Alice Bollenmiller, Andreas Wilhelm}
\VersuchLang{Realization of an native Android app using deep learning algorithms} %z.B. Ein Wichtiger Versuch
\VersuchKurz{Deep learning} %z.B. EWF
\Fach{Deep learning -  Dog Breed Classification} %z.B. Physik
\Studiengruppe{IG}
\Semester{WS 17/18}
\Betreuer{Lorenzo Servadei}
\Datum{\today} %falls gewünscht auf Versuchsdatum ändern, z.B. 21. November 2012

\PDFStartpage{1} %Seite die im Reader beim Start geöffnet wird
\MyParLineSkip{0.5} %Höhe der Absätze die durch \mypar erzeugt werden: 0...1

\praeinit %Initialisierung der Styles Teil 1

%umschalter zwischen WORKMODE und FINALMODE
%im Workmode gibts kein titelblatt und kein inhaltsverzeichnis,
%sodass man sich auf die arbeit an sich konzentrieren kann,
%und das setzen schneller geht. 
%wichtig: immer mindestens dreimal setzen lassen,
%wenn auf final umgeschaltet wird, 
%damit Veweise und Inhaltsverzeichnis stimmen!!!!
\setboolean{finalmode}{true}

%Umschalter zwischen draft und normal mode
%bewirkt, dass falls eingeschaltet, alle grafiken durch rahmen
%der entsprechenden größe ersetzt und dargestellt werden.
%erhöht die performance beim setzen deutlich und verhindert ablenkung 
%beim arbeiten durch bilder.
\setboolean{draftgraphics}{false}

\begin{document}
\postinit %Initialisierung der Styles Teil 2
%%AB HIER GEHT DIE ARBEIT LOS!

%\mysection{Introduction}
	\mysubsection{Deep learning}
In this chapter, a short introduction leads to the subject of Deep learning. Furthermore, the scope of this work and its points of referencee are described and localized. 
	\mysubsection{Terms of Referencee}
		
In 2015, AlphaGo - a computer program developed by Google's DeepMind Group that was trained to play the strategic board game Go - was the first program to defeat the multiple European champion Fan Hui under tournament conditions. Back then, terms like Artifical Intelligence (AI), Machine Learning and Deep Learning were talked of, because those were the reason why a computer program was able to defeat a human being. One of the most frequent used techniques of AI is Machine Learning. It uses algorithms to parse data, process it and learn from it. The result is a prediction or determination as a conclusion of what was learnt from the dataset. Deep Learning - which is part of the Machine Learning techniques - sells its application particularly in the field of language and image processing. \\

In 1958, Frank Rosenblatt introduced the concept of the perceptron which is the fundamental idea of all Deep learning approaches. It consists of multiple artifical neurons which are coupled with weights and biases. In the case of a single-layer perceptron, the input nodes are fully connected to one or more output nodes. During the learning process the weights are adapted according to the learning progress. When such a structure is extended with layers, it becomes a multi-layer perceptron (MLP). This basic structure can be found in special neuronal network architectures e.g. Convolutional Neuronal Networks (CNN). CNNs which are frequently used for object detection and image/ audio recognition etc. consists of multiple neurons. When a neuron receives an input, it performs a dot production and adapts the weights. Because of their special structure, CNNs are able to detect local properties of an image. Basically, the network represents a differentiable score function which is applied on the raw image pixels and computes the class scores as an output.
 
	\mysubsection{Terms of Referencee}
The problem of Deep learning architectures is their high performance requirements regarding computational power. Common frameworks for open source development in the field of deep learning which are discussed in ---- introduced several models for the integration in mobile applications. \\

In order to overcome the above mentioned problem, optimizing methods will be combined with appropriate models which require less computing power and are suitable for mobile application integration. As a result of this work, a dog breed analyzer will be implemented. This mobile app will take a live stream as an input and determine the breed of the focused dog. The three highest probabilities of breeds will be shown by the app.

	

%Example: Paper Outline
%    Introduction
%        Statement of the Problem
%        Definition of Terms
%        Theoretical Framework
%        Methodology
%            Type of Research
%            Respondents
%            Questionnaire
%        Hypothesis
%        Review of Related Literature
%        Scope and Limitations
%        Significance of the Study
%    Body
%        Background of the Study
%            Benefits of Breastfeeding
%            WHO Recommendations
%            The International Code of Marketing of Breast Milk Substitutes
%            The Baby-Friendly Hospital Initiative
%            The Innocenti Declaration on the Protection, Promotion and Support of Breastfeeding
%            National Situationer
%            The Milk Code
%            BFHI in the Philippines
%            Milk Code Violations
%            Formula Feeding
%            Factors Influencing the Decision Regarding Infant Feeding Method
%            Area Situationer
%        Presentation and Analysis of Data
%            Socio-economic Demographic Profile of Mothers
%            Information Regarding Current (Youngest) Infant
%            Current Infant Feeding Practices of Mothers
%                Exclusive Breastfeeding
%                Mixed Feeding
%                Formula Feeding
%            Previous Infant Feeding Practices
%            Maternal Knowledge
%            Correlation Tests
%    Conclusion
%        Concluding Statement
%            Analytical Summary
%            Thesis Reworded
%        Recommendations

\section{Introduction}
	\subsection{Deep learning}
		- what is deep learning -> purpose, usage, current research projects, state of the arts
	\subsection{Terms of Referencee}
		- dog breed analyzer -> goals, purpose, \\
		-> high perfomance computing but native android app
		

\section{Methodological fundamentals}
	\subsection{Common Frameworks for Deep Learning Applications}
		- some examples, tensorflow (tensorflow slim -> High  level api for easier use, tensorflow lite), Caffe, Keras, Torch, PyTorch, ... \\
		https://datahub.packtpub.com/deep-learning/top-10-deep-learning-frameworks/ \\
	\subsection{Common Models in Deep Learning Applications}
		- short differences between different architecuteres (?, CNN, RNN) \\
 		- AlexNet, Mobilenet, Inception, VGG, -> short decsription, useCases, important things, differences \\
	\subsection{Qualified Models for mobile App Integration}
 		- Mobilenet, Inception etc -> short decsription, useCases, important things, differences \\
	\subsection{Key requirements for an appropriate dataset}
		- generall why you need a huge dataset -> different backgrounds \\
		- self trained needs a huge dataset, a lot of computing performance and time \\
		-> so use pre trained, if small dataset. \\
		-> pretrained used millions of pictures (e.g. ImageNet)

\section{Concept}
	\subsection{Frameworks}
	- tensorflow -> why	
	\subsection{Model based Architectures}
	 - general architectures of models -> Mobilenet, Inception
	\subsection{Application based Architecture}

\section{Realisation}
	\subsection{dataset}
	\subsection{hardware environment}
		used CPU, GPU -> NVIDIA, handys
	\subsection{software environment}
	- Bazel, Java, Android Studio, Python, Operating System \\
 	- Android system
	\subsection{installation of software}
 		- software environment
			\subsubsection{Tensorflow based on Python}
			\subsubsection{Tensorflow based on Bazel}
				- e.g. Workspace changes for Android SDK, msse4.2
			\subsubsection{Installing Android Studio and its Delevopment Kit}
				- also possible with bazel but easier Android studio (needs correct versions of sdk, ndk) \\
				- SDK, NDK \\
				- IMPORTANT: tf versions updaten (same as trained)

	\subsection{building the models}
	-> evtl extra subsubsection: \\
		- execution methods -> Bazel and Python (incompatible versions) \\
		- Mobilnet -> steps, optimierung \\
		- Inception -> steps, optimierung \\
		- time related differences of execution  \\
		  -> time CPUs/GPU

	\subsection{Output Tests and Validation}
	 	- test pictures and if it works -> label image \\
	 	- validation script?!

	\subsection{Implementation of an native Android App}
		- list all necessary things to do (e.g. tensorflow version, Interpreter -> load Model)

	\subsection{Deployment and Validation}

\section{Evaluation}
- prio von nierdig zu hoch \\
- regarding implementation time \\
- regarding performance \\
- regarding quality in accuracy \\
- handy perfomance? \\

\section{Conclusion}
- tutorials not complete, different \\
- which model is better \\
- prospects, improvements, Recommendations \\



%%HIER ENDET DIE ARBEIT! DER REST IST KOMMENTAR MIT EIN PAAR PRAKTISCHEN FORMATIERUNGEN!!
\end{document}



%%%%%EINIGE SEHR PRAKTISCHE FORMATIERUNGSBEFEHLE UM ALLES SCHÖN UND EINHEITLICH AUSSEHEN ZU LASSEN!!%%%%%%%
%%%%%%%%%%%%%%%%%%%%%%%%%%%%%%%%%%%%%%%%%%%%%%%%%%%%%%%%%%%%%%%%%%%%%%%%%%%%%%%%%%%%%%%%%%%%%%%%%%%%%%%%%%%
%%%%%%%%%%%%%%%%%%%%%COMMENT%%%%%%%%%%%%%%%%%%%%%%COMMENT%%%%%%%%%%%%%%%%%%%%%%COMMENT%%%%%%%%%%%%%%%%%%%%%
%%%%%%%%%%%%%%%%%%%%%%%%%%%%%%%%%%%%%%%%%%%%%%%%%%%%%%%%%%%%%%%%%%%%%%%%%%%%%%%%%%%%%%%%%%%%%%%%%%%%%%%%%%%
\begin{comment}

% FORMELN UND PARAMETERBESCHREIBUNGEN
\begin{align}
\Delta \varphi = 360^\circ \cdot \frac{\Delta t}{T} = 360^\circ \cdot \Delta t \cdot f
\end{align}
wobei:
\begin{conditions} % *sternchen für zeilenübergreifende beschreibungstexte!!!
\Delta \varphi	& Phasenverschiebung \\
\Delta t		& Zeitdifferenz zwischen Eingangsspannung und gedämpfter Ausgangsspannung\\
T				& Periodendauer der Eingangsspannung \\
f				& Frequenz der Eingangsspannung
\end{conditions}

% Aufrechte griechische Buchstaben für Einheiten
\upmu
\upalpha
...etc

% BILDER EINFÜGEN
\begin{figure}[h] %t=top b=bottom h=here p =eigene page
\centering
\includegraphics[width=15cm]{media/hohlleiter}
\caption{Versuchsaufbau Hohlleiter}
\label{fig:label}
\end{figure}

% MEHRERE BILDER NEBENEINANDER
\begin{figure}[h] %t=top b=bottom h=here p =eigene page 
\flushright
\subfloat[Dispersionsrelation]{\includegraphics[width=7cm]{media/dispers}}
\subfloat[Phasen- und Gruppengeschwindigkeit]{\includegraphics[width=7cm]{media/phasgrupp}}
\end{figure}

%BILDER VOM TEXT UMFLOSSEN
\begin{wrapfigure}[10]{r}{8cm}
\centering
\includegraphics[width=7cm]{media/bragg}
\caption{\textsl{Aufbau eines Bragg-Spektrometers}}
\label{fig:bragg}
\end{wrapfigure}

%TABELLEN
\begin{table}[h]
\centering
\begin{tabular}{|c|c|}
\hline
$z$ [mm] & $\Delta z$ [mm]\\
\hline
3,3 & 0 \\
4,9 & 1,6 \\
6,5 & 1,6 \\
8,0 & 1,5 \\
\hline
\end{tabular}
\caption{Position der Minima zueinander und jeweiliger gemessener Abstand}
\label{tab:label}
\end{table}

% VERBUNDENE ZELLEN IN TABELLEN
\begin{table}[h]
\centering
\begin{tabular}{|c|c|c|c|p{1cm}p{1cm}p{1cm}p{1cm}p{1cm}p{1cm}p{1cm}|}
\hline
A & B & C & D & \multicolumn{7}{|c|}{F}  \\ \hline
\multirow{ 2}{*}{1} & 0 & 6 & 230 & 35 & 40 & 55 & 25 & 40 & 35 & \\
& 1 & 5 & 195 & 25 & 50 & 35 & 40 & 45 &  &  \\ \hline
\end{tabular}
\caption{A test caption}
\label{tab:table2}
\end{table}

% MEHRERE TABELLEN NEBENEINANDER
\begin{table}[h]
\centering

\subfloat[PVC, $d=15,0\,$mm]{
\begin{tabular}[b]{|c|c|c|}
\hline
mit Platte & ohne Platte & $\Delta s$\\
\hline
3,6 & 4,85 & 1,25\\
3,55 & 4,8 & 1,25\\
3,5 & 4,75 & 1,25\\
3,55 & 4,85 & 1,3\\
\hline
\end{tabular}
}

\subfloat[PE, $d=12,2\,$mm]{
\begin{tabular}[b]{|c|c|c|}
\hline
mit Platte & ohne Platte & $\Delta s$\\
\hline
3,5 & 4,2 & 0,7\\
3,5 & 4,25 & 0,75\\
3,5 & 4,2 & 0,7\\
\hline
\end{tabular}
}

\caption{Materialproben und zugehörige Messwerte in [cm]}
\end{table}

% FLOATS ERZWINGEN = TABELLEN UND BILDER ZWINGEND EINFÜGEN
\FloatBarrier

% ABSÄTZE MIT EINSTELLBARER UND NACHTRÄGLICH GLOBAL ÄNDERBARER DISTANZ
\mypar

% PDFs ANHÄNGEN
\includepdf[pages=-]{media/protokoll}

%BILDER UND TABELLEN REFERENZIEREN
\ref{labelname}
\pageref{labelname}

%REFERENZIERUNGSREGELN ZUR ÜBERSICHTLICHKEIT
bilder: fig:label
tabellen: tab:label
gleichungen: eq:label

\end{comment}
%%%%%%%%%%%%%%%%%%%%%%%%%%%%%%%%%%%%%%%%%%%%%%%%%%%%%%%%%%%%%%%%%%%%%%%%%%%%%%%%%%%%%%%%%%%%%%%%%%%%%%%%%%%
%%%%%%%%%%%%%%%%%%%%%COMMENT%%%%%%%%%%%%%%%%%%%%%%COMMENT%%%%%%%%%%%%%%%%%%%%%%COMMENT%%%%%%%%%%%%%%%%%%%%%
%%%%%%%%%%%%%%%%%%%%%%%%%%%%%%%%%%%%%%%%%%%%%%%%%%%%%%%%%%%%%%%%%%%%%%%%%%%%%%%%%%%%%%%%%%%%%%%%%%%%%%%%%%%
