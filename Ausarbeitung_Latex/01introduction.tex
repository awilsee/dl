\mysection{Introduction}
In this chapter, a short introduction leads to the subject of Deep learning. Furthermore, the scope of this work and its points of referencee are described and localized.

	\mysubsection{Deep learning}
		
In 2015, AlphaGo - a computer program developed by Google's DeepMind Group that was trained to play the strategic board game Go - was the first program to defeat the multiple European champion Fan Hui under tournament conditions \citep{DeepMind}. Back then, terms like Artifical Intelligence (AI), Machine Learning and Deep Learning were talked of, because those were the reason why a computer program was able to defeat a human being. One of the most frequent used techniques of AI is Machine Learning. It uses algorithms to parse data, process it and learn from it. The result is a prediction or determination as a conclusion of what was learnt from the dataset \citep{NVIDIA-MichaelCopeland2016}. Deep Learning - which is part of the Machine Learning techniques - sells its application particularly in the field of language and image processing \citep{Wick2017}. \\

In 1958, Frank Rosenblatt introduced the concept of the perceptron which is the fundamental idea of all Deep learning approaches. It consists of multiple artifical neurons which are coupled with weights and biases. In the case of a single-layer perceptron, the input nodes are fully connected to one or more output nodes. During the learning process the weights are adapted according to the learning progress. When such a structure is extended with layers, it becomes a multi-layer perceptron (MLP). This basic structure can be found in special neural network architectures e.g. Convolutional Neural Networks (CNN). CNNs which are frequently used for object detection and image/ audio recognition etc. consists of multiple neurons \citep{Wick2017}. When a neuron receives an input, it performs a dot production and adapts the weights. Because of their special structure, CNNs are able to detect local properties of an image. Basically, the network represents a differentiable score function which is applied on the raw image pixels and computes the class scores as an output \citep{UniversityOfStanford}.
 
	\mysubsection{Terms of Referencee}
The problem of Deep learning architectures is their high performance requirements regarding computational power. Common frameworks for open source development in the field of deep learning which are discussed in \subsecref{Common Frameworks for Deep Learning Applications} introduced several models for the integration in mobile applications. \\

In order to overcome the above mentioned problem, optimizing methods will be combined with appropriate models which require less computing power and are suitable for mobile application integration. As a result of this work, a dog breed analyzer will be implemented. This mobile app will take a live camera stream as an input and determine the breed of the focused dog. The three highest probabilities of breeds will be shown by the app.

	\mysubsection{Recent Work GitHub Repository}
The entire work done within this paper can be found in the GitHub repository \url{https://github.com/awilsee/dl.git}.