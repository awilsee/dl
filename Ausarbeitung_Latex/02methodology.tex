\mysection{Methodological fundamentals}
This chapter describes the most frequently used frameworks in deep learning for developing applications. Furthermore, common models for deep learning are introduced followed by suitable models for mobile integration. The chapter closes listing key requirements for an appropriate dataset which increase the quality of the training results.

	\mysubsection{Common Frameworks for Deep Learning Applications}
The demands on neural networks increases with the complexity of problems to solve. Concurrently, there's an expanding offer of deep learning frameworks with a varity of features and tools. The most common used ones are represented in the following section.

		\mysubsubsection{Tensorflow}
In 2015, the Google Brain Team introduced the most popular deep learning API Tensorflow which is an open-source library for numerical computation. Its current version 1.4.1 was released on December 8th, 2017. Tensorflow is primarily used for machine learning and deep neural network research. Based on the programming language Python, Tensorflow is capable of running on multiple CPUs and GPUs. Furthermore, C++ and R are supported by Tensorflow. Another feature is the possibility to generate models and export them as .pb file which holds the graph definition (GraphDef). The export is done by protocol buffers (protobuf) which includes tools for serializing and processing structured data. When loading a .pb file by protobuf, an graph object is created which holds a network of nodes. Each of those nodes represent an operation and the output is used as input for another operation. This concept enables an user to create self-built tensors. 

		\mysubsubsection{Keras}
In order to simplify the utilization of Tensorflow the Python based interface Keras can be configured to work on top of Tensorflow. It allows building neural networks in a simple way and is part of Tensorflow.

		\mysubsubsection{Caffe}
Another deep learning library is Caffe which was developed by Berkeley Vision and Learning Center (BVLC). Based on C++ or Python, it focuses on modeling CNNs. An main advantage of Caffe is the offer of pretrained models available in the Caffe Model Zoo. 

		\mysubsubsection{Torch and PyTorch}
Besides Tensorflow and Caffe, Torch is another common deep learning framework. It was developed by Facebook, Twitter and Google. Based on C/C++, Torch supports CUDA for GPU processing. Like above mentioned frameworks, Torch facilitates the building of neural networks. The Python based version of Torch is available through PyTorch.
		
	\mysubsection{Common Models in Deep Learning Applications}
		- short differences between different architecuteres (?, CNN, RNN) \\
 		- AlexNet, Mobilenet, Inception, VGG, -> short decsription, useCases, important things, differences \\
	\mysubsection{Qualified Models for mobile App Integration}
 		- Mobilenet, Inception etc -> short decsription, useCases, important things, differences \\
 		
	\mysubsection{Key requirements for an appropriate dataset}
Supervised learning tasks such as image recognition are based on operations where an output is taken as an input for the next node. Every raw pixel input is taken to compute an intermediate representation - a vector containing all learned information about the dataset. As a consequence, the training results are only as good as the dataset itself. For better accuracy its important to train a model on a variety of images for each object which should later be classified by the model. It's recommended to take images of an object which were taken at different times, with different devices and at different places. Otherwise, the model will concentrate on other things like for example the background instead of details about the object itself. Therefore, a huge dataset is required especially for non pre-trained models. Training a model from scratch will require a huge dataset, a lot computing power and time. Whereas pre-trained models only require a small dataset of about hundreds of images. For that reason, a pre-trained model will be used in this work.