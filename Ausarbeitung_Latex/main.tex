\documentclass{scrartcl}
\usepackage{includes/mypack}
\usepackage{animate}
\usepackage{courier}
\usepackage{listings}
\usepackage[utf8]{inputenc}
\definecolor{mygreen}{rgb}{0,0.6,0}
\definecolor{mygray}{rgb}{0.5,0.5,0.5}
\definecolor{mymauve}{rgb}{0.58,0,0.82}
\lstset{ %
  backgroundcolor=\color{white},   % choose the background color; you must add \usepackage{color} or \usepackage{xcolor}; should come as last argument
  basicstyle=\footnotesize\ttfamily,        % the size of the fonts that are used for the code
  breakatwhitespace=false,         % sets if automatic breaks should only happen at whitespace
  breaklines=true,                 % sets automatic line breaking
  captionpos=b,                    % sets the caption-position to bottom
  commentstyle=\color{mygreen},    % comment style
  %deletekeywords={...},            % if you want to delete keywords from the given language
  %escapeinside={\%*}{*)},          % if you want to add LaTeX within your code
  extendedchars=true,              % lets you use non-ASCII characters; for 8-bits encodings only, does not work with UTF-8
  %frame=single,	                   % adds a frame around the code
  keepspaces=true,                 % keeps spaces in text, useful for keeping indentation of code (possibly needs columns=flexible)
  keywordstyle=\color{blue},       % keyword style
  language=python,                 % the language of the code
  %morekeywords={*,...},            % if you want to add more keywords to the set
  numbers=left,                    % where to put the line-numbers; possible values are (none, left, right)
  numbersep=5pt,                   % how far the line-numbers are from the code
  numberstyle=\tiny\color{mygray}, % the style that is used for the line-numbers
  rulecolor=\color{black},         % if not set, the frame-color may be changed on line-breaks within not-black text (e.g. comments (green here))
  showspaces=false,                % show spaces everywhere adding particular underscores; it overrides 'showstringspaces'
  showstringspaces=false,          % underline spaces within strings only
  showtabs=false,                  % show tabs within strings adding particular underscores
  stepnumber=1,                    % the step between two line-numbers. If it's 1, each line will be numbered
  stringstyle=\color{mymauve},     % string literal style
  tabsize=4,	                   % sets default tabsize to 2 spaces
  %title=\lstname                   % show the filename of files included with \lstinputlisting; also try caption instead of title
}


\Autoren{Alice Bollenmiller, Andreas Wilhelm}
\VersuchLang{Realization of an native Android app using deep learning algorithms} %z.B. Ein Wichtiger Versuch
\VersuchKurz{Deep learning} %z.B. EWF
\Fach{Deep learning -  Dog Breed Classification} %z.B. Physik
\Studiengruppe{IG}
\Semester{WS 17/18}
\Betreuer{Lorenzo Servadei}
\selectlanguage{english}
\Datum{\today} %falls gewünscht auf Versuchsdatum ändern, z.B. 21. November 2012

\PDFStartpage{1} %Seite die im Reader beim Start geöffnet wird
\MyParLineSkip{0.5} %Höhe der Absätze die durch \mypar erzeugt werden: 0...1

\praeinit %Initialisierung der Styles Teil 1

%umschalter zwischen WORKMODE und FINALMODE
%im Workmode gibts kein titelblatt und kein inhaltsverzeichnis,
%sodass man sich auf die arbeit an sich konzentrieren kann,
%und das setzen schneller geht. 
%wichtig: immer mindestens dreimal setzen lassen,
%wenn auf final umgeschaltet wird, 
%damit Veweise und Inhaltsverzeichnis stimmen!!!!
\setboolean{finalmode}{true}

%Umschalter zwischen draft und normal mode
%bewirkt, dass falls eingeschaltet, alle grafiken durch rahmen
%der entsprechenden größe ersetzt und dargestellt werden.
%erhöht die performance beim setzen deutlich und verhindert ablenkung 
%beim arbeiten durch bilder.
\setboolean{draftgraphics}{false}


% Zitierung und Literaturverzeichnis
%-------------------------------------------------------------------
\usepackage{apacite}		% Zitierstil
\usepackage{natbib} 		% Erweiterte zitiermöglichkeiten 
\usepackage[fixlanguage]{babelbib}		% Übersetzung des Literaturverzeichnises und der Zitate
\selectbiblanguage{english}
%
%
%Abbildungen, Tabellen Ref mit Name
%
\newcommand{\mychapter}[1]{\chapter{#1} \label{chap:#1}}
\newcommand{\mysection}[1]{\section{#1} \label{sec:#1}}
\newcommand{\mysubsection}[1]{\subsection{#1} \label{subsec:#1}}
\newcommand{\mysubsubsection}[1]{\subsubsection{#1} \label{subsubsec:#1}}
%
\newcommand*{\chapref}[1]{\chaptername~\ref{chap:#1}}
\newcommand*{\secref}[1]{section~\ref{sec:#1}}
\newcommand*{\subsecref}[1]{section~\ref{subsec:#1}}
\newcommand*{\subsubsecref}[1]{section~\ref{subsubsec:#1}}
\newcommand*{\listref}[1]{listing~\ref{list:#1}}
\newcommand*{\figref}[1]{\figurename~\ref{fig:#1}}
\newcommand*{\tabref}[1]{\tablename~\ref{tab:#1}}
\usepackage[format=plain,
      justification=raggedright,
      singlelinecheck=false]
     {caption}
     
% Colors
%-------------------------------------------------------------------
%usepackage[usenames]{color}
\definecolor{myMaroon}{rgb}{0.5,0.25,0}
\definecolor{gray}{gray}{0.9}
\definecolor{darkGray}{gray}{0.5}
\definecolor{lightGreen}{rgb}{0,0.2,0}
\definecolor{darkGreen}{rgb}{0,0.4,0}


% Listings Einbettung von Code (C/C++, Java, ...)
%-------------------------------------------------------------------
\usepackage{verbatim}		% Sorgt dafür, dass Text so dargestellt wird wie er eingegeben ist,
                            % es werden keine Leerzeichen oder Tabs entfernnt
\usepackage{moreverb}		% Erweiterte Möglichkeiten für verbatim                            
\usepackage{listings}		% Source-Code printer for LaTeX
\lstset{language=Python}    % Python
%\lstloadlanguages{[Visual]C++,[ISO]C++}
\lstset{numbers=left, numberstyle=\tiny, numbersep=-6pt,tabsize=3, stepnumber=1}
\lstset{framexleftmargin=-5mm, frame=single, rulesepcolor=\color{darkGray}}
\lstset{captionpos=b} 							% Beschriftung unter Listing
\lstset{backgroundcolor=\color{gray}}
\lstset{basicstyle=\scriptsize\ttfamily} 				% alle listings winzig drucken 
%\lstset{commentstyle=\color{green}} 			% Kommentare gruen drucken 
%\lstset{keywordstyle=\color{blue}\bfseries}		% Schlüsselwörter fett und blau
%\lstset{morekeywords={uint8_t, uint16_t, uint32_t, uint64_t, int8_t, int16_t, int32_t, int64_t,}}


\begin{document}
\postinit %Initialisierung der Styles Teil 2
%%AB HIER GEHT DIE ARBEIT LOS!

%\mysection{Introduction}
	\mysubsection{Deep learning}
In this chapter, a short introduction leads to the subject of Deep learning. Furthermore, the scope of this work and its points of referencee are described and localized. 
	\mysubsection{Terms of Referencee}
		
In 2015, AlphaGo - a computer program developed by Google's DeepMind Group that was trained to play the strategic board game Go - was the first program to defeat the multiple European champion Fan Hui under tournament conditions. Back then, terms like Artifical Intelligence (AI), Machine Learning and Deep Learning were talked of, because those were the reason why a computer program was able to defeat a human being. One of the most frequent used techniques of AI is Machine Learning. It uses algorithms to parse data, process it and learn from it. The result is a prediction or determination as a conclusion of what was learnt from the dataset. Deep Learning - which is part of the Machine Learning techniques - sells its application particularly in the field of language and image processing. \\

In 1958, Frank Rosenblatt introduced the concept of the perceptron which is the fundamental idea of all Deep learning approaches. It consists of multiple artifical neurons which are coupled with weights and biases. In the case of a single-layer perceptron, the input nodes are fully connected to one or more output nodes. During the learning process the weights are adapted according to the learning progress. When such a structure is extended with layers, it becomes a multi-layer perceptron (MLP). This basic structure can be found in special neuronal network architectures e.g. Convolutional Neuronal Networks (CNN). CNNs which are frequently used for object detection and image/ audio recognition etc. consists of multiple neurons. When a neuron receives an input, it performs a dot production and adapts the weights. Because of their special structure, CNNs are able to detect local properties of an image. Basically, the network represents a differentiable score function which is applied on the raw image pixels and computes the class scores as an output.
 
	\mysubsection{Terms of Referencee}
The problem of Deep learning architectures is their high performance requirements regarding computational power. Common frameworks for open source development in the field of deep learning which are discussed in ---- introduced several models for the integration in mobile applications. \\

In order to overcome the above mentioned problem, optimizing methods will be combined with appropriate models which require less computing power and are suitable for mobile application integration. As a result of this work, a dog breed analyzer will be implemented. This mobile app will take a live stream as an input and determine the breed of the focused dog. The three highest probabilities of breeds will be shown by the app.

	

%Example: Paper Outline
%    Introduction
%        Statement of the Problem
%        Definition of Terms
%        Theoretical Framework
%        Methodology
%            Type of Research
%            Respondents
%            Questionnaire
%        Hypothesis
%        Review of Related Literature
%        Scope and Limitations
%        Significance of the Study
%    Body
%        Background of the Study
%            Benefits of Breastfeeding
%            WHO Recommendations
%            The International Code of Marketing of Breast Milk Substitutes
%            The Baby-Friendly Hospital Initiative
%            The Innocenti Declaration on the Protection, Promotion and Support of Breastfeeding
%            National Situationer
%            The Milk Code
%            BFHI in the Philippines
%            Milk Code Violations
%            Formula Feeding
%            Factors Influencing the Decision Regarding Infant Feeding Method
%            Area Situationer
%        Presentation and Analysis of Data
%            Socio-economic Demographic Profile of Mothers
%            Information Regarding Current (Youngest) Infant
%            Current Infant Feeding Practices of Mothers
%                Exclusive Breastfeeding
%                Mixed Feeding
%                Formula Feeding
%            Previous Infant Feeding Practices
%            Maternal Knowledge
%            Correlation Tests
%    Conclusion
%        Concluding Statement
%            Analytical Summary
%            Thesis Reworded
%        Recommendations

\mysection{Introduction}
	\mysubsection{Deep learning}
In this chapter, a short introduction leads to the subject of Deep learning. Furthermore, the scope of this work and its points of referencee are described and localized. 
	\mysubsection{Terms of Referencee}
		
In 2015, AlphaGo - a computer program developed by Google's DeepMind Group that was trained to play the strategic board game Go - was the first program to defeat the multiple European champion Fan Hui under tournament conditions. Back then, terms like Artifical Intelligence (AI), Machine Learning and Deep Learning were talked of, because those were the reason why a computer program was able to defeat a human being. One of the most frequent used techniques of AI is Machine Learning. It uses algorithms to parse data, process it and learn from it. The result is a prediction or determination as a conclusion of what was learnt from the dataset. Deep Learning - which is part of the Machine Learning techniques - sells its application particularly in the field of language and image processing. \\

In 1958, Frank Rosenblatt introduced the concept of the perceptron which is the fundamental idea of all Deep learning approaches. It consists of multiple artifical neurons which are coupled with weights and biases. In the case of a single-layer perceptron, the input nodes are fully connected to one or more output nodes. During the learning process the weights are adapted according to the learning progress. When such a structure is extended with layers, it becomes a multi-layer perceptron (MLP). This basic structure can be found in special neuronal network architectures e.g. Convolutional Neuronal Networks (CNN). CNNs which are frequently used for object detection and image/ audio recognition etc. consists of multiple neurons. When a neuron receives an input, it performs a dot production and adapts the weights. Because of their special structure, CNNs are able to detect local properties of an image. Basically, the network represents a differentiable score function which is applied on the raw image pixels and computes the class scores as an output.
 
	\mysubsection{Terms of Referencee}
The problem of Deep learning architectures is their high performance requirements regarding computational power. Common frameworks for open source development in the field of deep learning which are discussed in ---- introduced several models for the integration in mobile applications. \\

In order to overcome the above mentioned problem, optimizing methods will be combined with appropriate models which require less computing power and are suitable for mobile application integration. As a result of this work, a dog breed analyzer will be implemented. This mobile app will take a live stream as an input and determine the breed of the focused dog. The three highest probabilities of breeds will be shown by the app.

	

\mysection{Methodological fundamentals}
This chapter describes the most frequently used frameworks in deep learning for developing applications. Furthermore, common models for deep learning are introduced followed by suitable models for mobile integration. The chapter closes listing key requirements for an appropriate dataset which increase the quality of the training results.

	\mysubsection{Common Frameworks for Deep Learning Applications}
The demands on neural networks increases with the complexity of problems to solve. Concurrently, there's an expanding offer of deep learning frameworks with a varity of features and tools. The most common used ones are represented in the following section.

		\mysubsubsection{Tensorflow}
In 2015, the Google Brain Team introduced the most popular deep learning API Tensorflow which is an open-source library for numerical computation. Its current version 1.4.1 was released on December 8th, 2017. Tensorflow is primarily used for machine learning and deep neural network research. Based on the programming language Python, Tensorflow is capable of running on multiple CPUs and GPUs. Furthermore, C++ and R are supported by Tensorflow. Another feature is the possibility to generate models and export them as .pb file which holds the graph definition (GraphDef). The export is done by protocol buffers (protobuf) which includes tools for serializing and processing structured data. When loading a .pb file by protobuf, an graph object is created which holds a network of nodes. Each of those nodes represent an operation and the output is used as input for another operation. This concept enables an user to create self-built tensors. 

		\mysubsubsection{Keras}
In order to simplify the utilization of Tensorflow the Python based interface Keras can be configured to work on top of Tensorflow. It allows building neural networks in a simple way and is part of Tensorflow.

		\mysubsubsection{Caffe}
Another deep learning library is Caffe which was developed by Berkeley Vision and Learning Center (BVLC). Based on C++ or Python, it focuses on modeling CNNs. An main advantage of Caffe is the offer of pretrained models available in the Caffe Model Zoo. 

		\mysubsubsection{Torch and PyTorch}
Besides Tensorflow and Caffe, Torch is another common deep learning framework. It was developed by Facebook, Twitter and Google. Based on C/C++, Torch supports CUDA for GPU processing. Like above mentioned frameworks, Torch facilitates the building of neural networks. The Python based version of Torch is available through PyTorch.
		
	\mysubsection{Common Models in Deep Learning Applications}
		- short differences between different architecuteres (?, CNN, RNN) \\
 		- AlexNet, Mobilenet, Inception, VGG, -> short decsription, useCases, important things, differences \\
	\mysubsection{Qualified Models for mobile App Integration}
 		- Mobilenet, Inception etc -> short decsription, useCases, important things, differences \\
 		
	\mysubsection{Key requirements for an appropriate dataset}
Supervised learning tasks such as image recognition are based on operations where an output is taken as an input for the next node. Every raw pixel input is taken to compute an intermediate representation - a vector containing all learned information about the dataset. As a consequence, the training results are only as good as the dataset itself. For better accuracy its important to train a model on a variety of images for each object which should later be classified by the model. It's recommended to take images of an object which were taken at different times, with different devices and at different places. Otherwise, the model will concentrate on other things like for example the background instead of details about the object itself. Therefore, a huge dataset is required especially for non pre-trained models. Training a model from scratch will require a huge dataset, a lot computing power and time. Whereas pre-trained models only require a small dataset of about hundreds of images. For that reason, a pre-trained model will be used in this work.

\mysection{Concept}
First, this chapter describes the selection of the appropriate framework. Futhermore, the structure of the model which was used for classification is explained based on its architecture. The chapter closes with the class diagram of the mobile application.

	\mysubsection{Frameworks}
As a deep learning framework Tensorflow was used to retrain the model. This decision was mainly based on recommendations. Even companies like e.g. NVIDIA Corporation, Intel Corporation etc. use Tensorflow. It is one of the common frameworks for deep learning applications and also provides solutions for integration in mobile apps. Furthermore, Tensorflow provides a variety of tutorials for working with neural networks. Beside those advantages, there is a large community about Tensorflow talking about issues and solutions. \\

To run the tensor within a mobile application, the first approach was to use Tensorflow Lite which is still in development state. But many attempts resulted in corrupt models which caused the app to terminate. Because of this experiences Tensorflow Mobile was used to optimize the model for app integration. 
		
	\mysubsection{Qualified models for mobile app integration}
In Tensorflow Lite only InceptionV3 and MobileNet models are supported. With Tensorflow Mobile it's the same. There are different versions of MobileNet. They differ in the input image size and in the number of parameters which is also proportional to the size and needed computation power of the network.

While the Inception model was well known and established, the MobileNet is relatively new. InceptionV3 models gets an higher accuracy than MobileNets but MobileNets are more optimized on small size, low latency and low-power consumption which are important characteristics for mobile usage. \citep{TensorFlowMobileNet} \\


In \figref{FH-Logo8} the MobileNet and Inception is compared with each other and also with other popular models already mentioned before in \subsecref{Common Models in Deep Learning Applications}. It shows the Top-1 accuracy and the Multiply-Accumulates (MACs) which measures the number of fused Multiplication and Addition operations. The latter numbers reflect the latency and power consumption of the network and so in result the efficiency. For well comparison reasons between MobileNet and Inception, a MobileNet model with similiar accuracy to the Inception one was picked, to be more precisely the MobileNet\_1.0\_224. To check out the possibilities of MobileNet, another model, the MobileNet\_0.50\_244, was selected, too.

\begin{figure}[htbp]
\includegraphics[width=0.8\textwidth]{includes/MUASlogo}
\caption[Comparison of popular models]{Comparison of popular models \citep{TensorFlowMobileNet}}
\label{fig:FH-Logo8}
\end{figure} 

	\mysubsection{Application based Architecture} Alice
This chapter describes the functions to a basic understanding of how the application works based on the application's architecture. It focuses on the important classes and methods of the mobile application. \\
Tensorflow provides a mobile application for demo purposes. Because of the complexity the application was adapted to our needs. The approach was to understand how the application was implemented regarding the functionality of the camera, image input for the network and background classification besides live camera stream. \\

The ClassifierActivity is set as the launcher activity of the application. It extends the class CameraActivity and loads the CameraFragment which controls the camera view. The CustomTextureView enables the possibility to capture frames from a camera stream and process it. Because of this function, the CustomTextureView is part of the class CameraFragment. If an image is captured, the method onImageAvailable of the class ClassifierActivity is invoked. The classification itself is done by the TensorflowImageClassifier which implements the interface Classifier from the Tensorflow API. The results are displayed by the class RecognitionScoreView which implements the interface ResultsView.

\mysection{Realization}
In general, this chapter describes the methodical procedure of solving the above mentioned problem \subsecref{Terms of Referencee}. After describing the used dataset all required software and hardware components are explained in detail. Furthermore, the chapter leads through the installation steps of Tensorflow and the setup of Android Studio. Followed by the installation process the retraining of a pre-trained model is depicted. Afterwards, the re-trained model is tested and validated. The chapter ends with the description of the realization of the Android app.

	\mysubsection{dataset} Andi
	\mysubsection{hardware environment} Andi
		used CPU, GPU -> NVIDIA, handys

	\mysubsection{installation of software} Andi
This chapter includes all necessary steps for installing the software environment including Tensorflow.

 			\mysubsubsection{Prerequisites}
The software environment was set up on the Linux distribution Ubuntu 16.04 LTS. To install the software environment for Tensorflow Python is required. Therefore, the current version of Python 3.6 was installed by default. Tensorflow also supports Bazel which was installed by following command.

\begin{lstlisting}[caption=Bazel Installation, label=list:bazel, language=bash]
	sudo apt-get install openjdk-8-jdk
	
	echo "deb [arch=amd64] http://storage.googleapis.com/bazel-apt stable jdk1.8" | sudo tee /etc/apt/				sources.list.d/bazel.list
	curl https://bazel.build/bazel-release.pub.gpg | sudo apt-key add -
	
	sudo apt-get update && sudo apt-get install bazel
	
	sudo apt-get upgrade bazel
\end{lstlisting}	

Futhermore, the package and environment management tool Anaconda was installed by the following steps:
First, the Anaconda installer was downloaded from \url{https://www.anaconda.com/download/#linux}. During the installation process the prompts were answered by the default suggestions except the following prompt: "Do you wish the installer to prepend the Anaconda3 install location to PATH in your /home/aw/.bashrc ? [yes|no]". "yes" was typed in and conda was tested using the "conda list" command. \\

As an app development environment the free IDE Android Studio in its version 3.0.1 was installed by downloading from \url{https://developer.android.com/studio/index.html}, extracting and following the instruction steps. Required dependencies are installed by Android Studio itself. So, the SDK in the version 26.1.1 was used. \\

In the development of native Android Apps Java is used as the programming language. Within the installation of Anaconda, the JDK in the version 8 was installed.
- test anaconda, tensorflow environment look where it belongs

 	- CUDA, CUDNN
 	
			\mysubsubsection{Tensorflow based on Python}

			\mysubsubsection{Tensorflow based on Bazel}
				- e.g. Workspace changes for Android SDK, msse4.2
			\mysubsubsection{Installing Android Studio and its Delevopment Kit}
				- also possible with bazel but easier Android studio (needs correct versions of sdk, ndk) \\
				- SDK, NDK \\
				- IMPORTANT: tf versions updaten (same as trained)

	\mysubsection{building the models} Alice bis Steps, Andi ab Optimierung, time GPU/CPU
	-> evtl extra subsubsection: \\
		- execution methods -> Bazel and Python (incompatible versions) \\
		- Mobilnet -> steps, optimierung \\
		- Inception -> steps, optimierung \\
		- time related differences of execution  \\
		  -> time CPUs/GPU

	\mysubsection{Output Tests and Validation} Andi
	 	- test pictures and if it works -> label image \\
	 	- validation script?! --> in Evaluierung

	\mysubsection{Implementation of an native Android App} Alice
This subchapter describes the processing of the camera input stream and classifying the particular images. Because of the vast extent of the application, the focus lies on the explanation of how the app works and on its important implementation parts. \\

When click on the icon of the app, a camera view showing results in the bottom part is loaded. In the background, the ClassifierActivity is set as the launcher activity in the AndroidManifest.xml which is a file containing all configuration settings for the app. When an activity is loaded for the first time, the onCreate method is invoked. Because the ClassifierActivity extends the CameraActivity the onCreate method of the CameraActivity is executed and loads the layout activity_camera.xml consisting of a container and a result field. Next, the setFragment method of the class CameraActivity is invoked if permission for the camera is given. This causes the replacement of the container with the adapted fragment according to the actual size and orientation of the screen. When the size is set, the method onPreviewSizeChosen is invoked where a TensorFlowImageClassifier is instantiated with following input parameters: MODEL_FILE, LABEL_FILE, INPUT_SIZE, IMAGE_MEAN, IMAGE_STD, INPUT_NAME, OUTPUT_NAME. Details about the Classification follow later in the section.\\

When the fragment was replaced by the CameraActivity, an instance of the class CameraFragment was created. Next, the method onViewCreated of the class CameraFragment is called and a texture view is created. The latter class is part of the Android API and used for frame capturing from an image stream as OPENGL ES texture. Thereby, the most recent image is stored within a texture object. This object is observed by a listener which invokes the method onSurfaceTextureAvailable when the texture object is available. Within this method the camera is opened by the method openCamera(width, height) given the width and height of the camera preview \citep{AndroidDevelopers}. First, the method openCamera sets the camera parameters within setUpCameraOutputs e.g. sensorOrientation, previewSize, cameraId etc. In order to classifiy a given OPENGL ES texture, the coordinate column vectors of the texture must be transformed into the proper sampling location in the streamed texture. So, the matrix has to be prepared with the correct configuration which happens in the method configureTransform. The next important step is done by the camera manager which opens the camera by invoking the method openCamer(cameraId, stateCallback, backgroundHandler). The id represents the specific camera device whereas the stateCallback is necessary to manage the life circle of the camera device and to handle different states of the camera device. The backgroundHandler ensures that the classification is done in background mode. When the camera is opened, the camera preview is started by the method createCameraPreviewSession. Within this method the preview reader is initialized and set to read images from the ImageListener which observes whether an image is available. Therefore, an Imagelistener which was instantiated by the CameraActivity is transfered to the CameraFragment. \\

When an image is available the method onImageAvailable of the class ClassifierActivity is invoked. The image is read by the preview reader and stored in an object. First, the image is preprocessed meaning transfered into planes, stored as bytes, cropped and stored as a Bitmap which is an Android graphic. Google provides an Tensorlflow Mobile API to run tensors on a performance critical device such as mobile devices. To use this API, the depedency has to be added to the build.gradle file \listref{tensorflow_api}. 

\begin{lstlisting}[caption=Tensorflow API in build.gradle, label=list:tensorflow_api, language=java]

	dependencies {
                compile 'org.tensorflow:tensorflow-android:+'
    }
\end{lstlisting}

When including the API, Tensorflow's class Classifier can be used to classify images. Therefore, the TensorFlowImageClassifier which was initialized before invokes the method recognizeImage(croppedBitmap) on the preprocessed image. Within this method, the image data is transfered from int to float. Then, the previous initialized object of the TensorflowInferenceInterface calls the method feed to pass the float values to the input layer of the model. Afterwards, the inferenceInterface invokes the method run with the specified output layer in order to process the classification. Subsequently, the output data is fetched by calling fetch on the output layer \listref{classify_android}.

\begin{lstlisting}[caption=Classifying images by the inferenceInterface, label=list:classify_android, language=java]

    // Copy the input data into TensorFlow.
    Trace.beginSection("feed");
    inferenceInterface.feed(inputName, floatValues, 1, inputSize, inputSize, 3);
    Trace.endSection();

    // Run the inference call.
    Trace.beginSection("run");
    inferenceInterface.run(outputNames, logStats);
    Trace.endSection();

    // Copy the output Tensor back into the output array.
    Trace.beginSection("fetch");
    inferenceInterface.fetch(outputName, outputs);
    Trace.endSection();
\end{lstlisting}

Afterwards, the results are reordered according to the highest probability and returned to the results field for displaying the output.
	\mysubsection{Deployment and Validation} Alice


\mysection{Evaluation} Andi \\
- prio von nierdig zu hoch \\
- regarding implementation time \\
- regarding performance \\
- regarding quality in accuracy \\
- handy perfomance? \\



\mysection{Conclusion}
At the beginning, a lot of research was done to answer the general question of 'how can a network be integrated in a mobile application and run on a performance critical device'. During the research, Tensorflow Lite was chosen as an API which enables running a network on an Android mobile phone. Because Tensorflow supports the models MobileNet and Inception specifically for mobile application usage, both were chosen. Based on this decision, both models were investigated and a comparison of both of them was incorporated into this work. After determining which framework and models are suitable for mobile applicaton support, the installation began. Because of a good documentation the installation of Tensorflow and its necessary dependencies was done quickly. Following the tutorials the models were retrained on dog images. For a first, the optimization of the models to its full degree was omitted and the focus was to include the optimized models in the mobile application for running. Therefore, Tensorflow Lite was used to convert the .pb file to a .tflite file. The conversion was conducted successfully, but the produced .tflite file caused the app to terminate. After proving the versions of all used APIs and reconverting the models several times, the app still collapsed without throwing an error. Even after optimizing, rounding, quantizing the models with different commands, the app failed to run. In addition, the tutorials and documentation are kept short on this subject. Because of this behaviour, Tensorflow Mobile was used instead of Tensorflow Lite which is still in development mode. After integration of the .pb file and its corresponding label file, the Tensorflow version had to be adapted. If adding '+' in the dependency section for Tensorflow instead of the specified version, Android Studio installs the most recent one. Futhermore, after integrating an InceptionV3 model, no results were displayed while the app was running. Therefore, the threshold was adapted to a lower value in order to display low results from the network. Later, the threshold was set to 0.1f and the models were optimized. Next, the models were optimized to their full degree based on varying the learning rate in relation to the training steps. Afterwards, the models were comprised and loaded into the mobile application. The result are two applications, one including MobileNet and another containing InceptionV3. In the last step, the models were compared based on performance, time expenditure for producing an optimized model and quality in accuracy. Especially for the time related evaluation, marker were placed in the scripts to measure the time needed for creating the bottlenecks, training on images and evaluating an image. The expectation was that the InceptionV3 was the most accurate model, but the one with the most required performance classifying an image in the mobile application. Futhermore, the MobileNet 1.0 was expected to be the most fastest and accurate model, whereas the MobileNet 0.50 might be accurate, but slower than its successor. It turned out contrary to expectations that the MobileNet 0.50 is the most accurate one with lowest performance required (refer to \secref{Evaluation}). As a last surprise, the app was running more smoothly on the Samsung S4 device than on the newer Motorola Moto X which contains a better processor.
\\
\\
\\

- tutorials not complete, different \\
- which model is better \\
- Tensorflow Lite conversion failed completely \\
- prospects, improvements, Recommendations \\
\\
\\ \\
Beispiele fürs referenzieren: \\
In \figref{FH-Logo} ist das HS München Logo zu sehen.

\begin{figure}[htbp]
\includegraphics[width=0.3\textwidth]{includes/MUASlogo}
\caption{FH-Logo}
\label{fig:FH-Logo}
\end{figure}

Oder auch eines Codes wie in  \listref{python_Code}.\\
\begin{lstlisting}[caption=Some python code, label=list:python_Code]

	bottleneck_path_2_bottleneck_values = {}


	def create_bottleneck_file(bottleneck_path, image_lists, label_name, index,
	                           image_dir, category, sess, jpeg_data_tensor,
	                           decoded_image_tensor, resized_input_tensor,
	                           bottleneck_tensor):
	  """Create a single bottleneck file."""
	  tf.logging.info('Creating bottleneck at ' + bottleneck_path)
	  image_path = get_image_path(image_lists, label_name, index,
	                              image_dir, category)
	  if not gfile.Exists(image_path):
	    tf.logging.fatal('File does not exist %s', image_path)
	  image_data = gfile.FastGFile(image_path, 'rb').read()
	  try:
	    bottleneck_values = run_bottleneck_on_image(
	        sess, image_data, jpeg_data_tensor, decoded_image_tensor,
	        resized_input_tensor, bottleneck_tensor)
	  except Exception as e:
	    raise RuntimeError('Error during processing file %s (%s)' % (image_path,
	                                                                 str(e)))
	  bottleneck_string = ','.join(str(x) for x in bottleneck_values)
	  with open(bottleneck_path, 'w') as bottleneck_file:
	    bottleneck_file.write(bottleneck_string)
\end{lstlisting}

Sectionrefs: In \secref{Methodological fundamentals} ist vieles noch nicht fertig. \\
SubSectionrefs: In \subsecref{Common Frameworks for Deep Learning Applications} wird dann näher auf den Inhalt eingegangen.\\
SubSubSectionrefs: In \subsubsecref{Tensorflow based on Python} gehts ans eingemachte.\\

Beispiele fürs zitieren:

Für einen noch besseren Überblick, kann das Buch von \citet{Butler2017} oder auch andere refs wie \citet{CS231nCNN} hinzugezogen werden \citep{Deeplearning4j2017}.\\
Wenn in klammern und Seitenzahl \citep[p. 3]{Butler2017}\\

als compared, aber ohne Seitenzahl \citep[cmp.][]{Butler2017} \\
als compared mit Seitenzahl, das nd heißt "no date", da keine Jahrezahl vorhanden \citep[cmp.][p. 5]{Wang} \\

%Beispiele
%\citet{goossens93} -> Goossens et al. (1993)
%\citep{goossens93} -> (Goossens et al., 1993)
%
%\citet*{goossens93} -> Goossens, Mittlebach, and Samarin (1993)
%\citep*{goossens93} -> (Goossens, Mittlebach, and Samarin, 1993)

\clearpage
\newcounter{anhang}
\pagenumbering{Roman}
%\bibliographystyle{gerapali} % autor, jahr gerapali
\bibliographystyle{apalike} % autor, jahr
\bibliography{references}
\addcontentsline{toc}{section}{Bibliography}
%\clearpage
\listoffigures
\addcontentsline{toc}{section}{List of figures}
\appendix
\newpage
\mysection{Appendix}

\mysubsection{Configure Bazel for Tensorflow}

\begin{lstlisting}[caption=Configure bazel for Tensorflow, label=list:configure_bazel, language=bash]
	$ cd tensorflow  # cd to the top-level directory created
	$ ./configure
	Please specify the location of python. [Default is /usr/bin/python]: /usr/bin/python3.6
	Found possible Python library paths:
	  /usr/local/lib/python3.6/dist-packages
	  /usr/lib/python3.6/dist-packages
	Please input the desired Python library path to use.  Default is [/usr/lib/python3.6/dist-packages]
	Using python library path: /usr/local/lib/python3.6/dist-packages
	Do you wish to build TensorFlow with MKL support? [y/N]
	No MKL support will be enabled for TensorFlow
	Please specify optimization flags to use during compilation when bazel option "--config=opt" is specified [Default is -march=native]:
	Do you wish to use jemalloc as the malloc implementation? [Y/n] n
	No jemalloc as malloc support will be enabled for TensorFlow.
	Do you wish to build TensorFlow with Google Cloud Platform support? [y/N] n
	No Google Cloud Platform support will be enabled for TensorFlow
	Do you wish to build TensorFlow with Hadoop File System support? [y/N] n
	No Hadoop File System support will be enabled for TensorFlow
	Do you wish to build TensorFlow with the XLA just-in-time compiler (experimental)? [y/N] n
	No XLA support will be enabled for TensorFlow
	Do you wish to build TensorFlow with Amazon S3 File System support? [Y/n]: n
	No Amazon S3 File System support will be enabled for TensorFlow.
	Do you wish to build TensorFlow with GDR support? [y/N]: n
	No GDR support will be enabled for TensorFlow.
	Do you wish to build TensorFlow with VERBS support? [y/N] n
	No VERBS support will be enabled for TensorFlow
	Do you wish to build TensorFlow with OpenCL support? [y/N] n
	No OpenCL support will be enabled for TensorFlow
	Do you wish to build TensorFlow with CUDA support? [y/N] Y
	CUDA support will be enabled for TensorFlow
	Please specify the Cuda SDK version you want to use, e.g. 7.0. [Leave empty to default to CUDA 8.0]: 8.0
	Please specify the location where CUDA 8.0 toolkit is installed. Refer to README.md for more details. [Default is /usr/local/cuda]:
	Please specify the cuDNN version you want to use. [Leave empty to default to cuDNN 6.0]: 6
	Please specify the location where cuDNN 6 library is installed. Refer to README.md for more details. [Default is /usr/local/cuda]:
	Please specify a list of comma-separated Cuda compute capabilities you want to build with.
	You can find the compute capability of your device at: https://developer.nvidia.com/cuda-gpus.
	Please note that each additional compute capability significantly increases your build time and binary size.
	[Default is: 5.2]: 5.2
	Do you want to use clang as CUDA compiler? [y/N] n
	nvcc will be used as CUDA compiler
	Please specify which gcc should be used by nvcc as the host compiler. [Default is /usr/bin/gcc]:
	Do you wish to build TensorFlow with MPI support? [y/N] n
	MPI support will not be enabled for TensorFlow
	Configuration finished
\end{lstlisting}

\begin{sidewaysfigure}
\mysubsection{Mobile Application Architecture in the form of a Class Diagram}
\includegraphics[width=0.95\textwidth]{includes/ClassDiagram}
\end{sidewaysfigure}
\newpage




\mysubsection{TensorBoard - optimal achieved accuracy of the different models}
\begin{figure}[htbp]
\centering
\includegraphics[width=0.95\textwidth]{includes/MobileNet05-700res}
\caption{TensorBoard - optimal achieved accuracy of MobileNet_0.5}
\label{fig:MobileNet05-700res}
\end{figure}

\begin{figure}[htbp]
\centering
\includegraphics[width=0.95\textwidth]{includes/MobileNet10-500res}
\caption{TensorBoard - optimal achieved accuracy of MobileNet_1.0}
\label{fig:MobileNet10-500res}
\end{figure}

\begin{figure}[htbp]
\centering
\includegraphics[width=0.95\textwidth]{includes/inception4000res}
\caption{TensorBoard - optimal achieved accuracy of InceptionV3}
\label{fig:inception4000res}
\end{figure}

\begin{figure}[htbp]
\centering
\includegraphics[width=0.95\textwidth]{includes/AllRes}
\caption{TensorBoard - optimal achieved accuracy of all models}
\label{fig:AllRes}
\end{figure}

\newpage



\mysubsection{Evaluation of optimization attempts}

\begin{table}[]
\centering
\begin{tabular}{|l|r|r|r|l}
\cline{1-4}
opt attempt & \multicolumn{1}{l|}{accuracy} & \multicolumn{1}{l|}{misclassified} & \multicolumn{1}{l|}{time to classify (App) {[}ms{]}} &  \\ \cline{1-4}
retrained & 0,566641033                   & 5                                           & 0                                &  \\ \cline{1-4}
opt1      & 0,566641033                   & 5                                           & 239,6                           &  \\ \cline{1-4}
opt2      & 0,487681977                   & 8                                           & 252,5                           &  \\ \cline{1-4}
opt3      & 0,566641033                   & 5                                           & 241,9                   &  \\ \cline{1-4}
opt4      & 0,483819713                   & 8                                           & 246,8                            &  \\ \cline{1-4}
\end{tabular}
\caption{MobileNet_0.5}
\label{tab:mobileNet05}
\end{table}

\begin{table}[]
\centering
\begin{tabular}{|l|r|r|r|l}
\cline{1-4}
opt attempt & \multicolumn{1}{l|}{accuracy} & \multicolumn{1}{l|}{misclassified} & \multicolumn{1}{l|}{time to classify (App) {[}ms{]}} &  \\ \cline{1-4}
retrained & 0,504834563                   & 7                                           & 0                                &  \\ \cline{1-4}
opt1      & 0,504834563                   & 7                                           & 316,2                            &  \\ \cline{1-4}
opt2      & 0,458024049                   & 8                                           & 310,2                           &  \\ \cline{1-4}
opt3      & 0,504834563                   & 7                                           & 323,4                            &  \\ \cline{1-4}
opt4      & 0,458474453                   & 8                                           & 320,6                           &  \\ \cline{1-4}
\end{tabular}
\caption{MobileNet_1.0}
\label{tab:mobileNet10}
\end{table}

\begin{table}[]
\centering
\begin{tabular}{|l|r|r|r|l}
\cline{1-4}
opt attempt & \multicolumn{1}{l|}{accuracy} & \multicolumn{1}{l|}{misclassified} & \multicolumn{1}{l|}{time to classify (App) {[}ms{]}} &  \\ \cline{1-4}
retrained   & 0,722748141                   & 2                                           & 0                                                    &  \\ \cline{1-4}
opt1        & 0,722748250                   & 2                                           & 1092,2                                              &  \\ \cline{1-4}
opt2        & 0,698508064                   & 3                                           & 1094,4                                               &  \\ \cline{1-4}
opt3        & 0,722748137                   & 2                                           & 1139,9                                               &  \\ \cline{1-4}
opt4        & 0,700055786                   & 3                                           & 1108,2                                               &  \\ \cline{1-4}
\end{tabular}
\caption{InceptionV3}
\label{tab:inception}
\end{table}

%%HIER ENDET DIE ARBEIT! DER REST IST KOMMENTAR MIT EIN PAAR PRAKTISCHEN FORMATIERUNGEN!!
\end{document}



%%%%%EINIGE SEHR PRAKTISCHE FORMATIERUNGSBEFEHLE UM ALLES SCHÖN UND EINHEITLICH AUSSEHEN ZU LASSEN!!%%%%%%%
%%%%%%%%%%%%%%%%%%%%%%%%%%%%%%%%%%%%%%%%%%%%%%%%%%%%%%%%%%%%%%%%%%%%%%%%%%%%%%%%%%%%%%%%%%%%%%%%%%%%%%%%%%%
%%%%%%%%%%%%%%%%%%%%%COMMENT%%%%%%%%%%%%%%%%%%%%%%COMMENT%%%%%%%%%%%%%%%%%%%%%%COMMENT%%%%%%%%%%%%%%%%%%%%%
%%%%%%%%%%%%%%%%%%%%%%%%%%%%%%%%%%%%%%%%%%%%%%%%%%%%%%%%%%%%%%%%%%%%%%%%%%%%%%%%%%%%%%%%%%%%%%%%%%%%%%%%%%%
\begin{comment}

% FORMELN UND PARAMETERBESCHREIBUNGEN
\begin{align}
\Delta \varphi = 360^\circ \cdot \frac{\Delta t}{T} = 360^\circ \cdot \Delta t \cdot f
\end{align}
wobei:
\begin{conditions} % *sternchen für zeilenübergreifende beschreibungstexte!!!
\Delta \varphi	& Phasenverschiebung \\
\Delta t		& Zeitdifferenz zwischen Eingangsspannung und gedämpfter Ausgangsspannung\\
T				& Periodendauer der Eingangsspannung \\
f				& Frequenz der Eingangsspannung
\end{conditions}

% Aufrechte griechische Buchstaben für Einheiten
\upmu
\upalpha
...etc

% BILDER EINFÜGEN
\begin{figure}[h] %t=top b=bottom h=here p =eigene page
\centering
\includegraphics[width=15cm]{media/hohlleiter}
\caption{Versuchsaufbau Hohlleiter}
\label{fig:label}
\end{figure}

% MEHRERE BILDER NEBENEINANDER
\begin{figure}[h] %t=top b=bottom h=here p =eigene page 
\flushright
\subfloat[Dispersionsrelation]{\includegraphics[width=7cm]{media/dispers}}
\subfloat[Phasen- und Gruppengeschwindigkeit]{\includegraphics[width=7cm]{media/phasgrupp}}
\end{figure}

%BILDER VOM TEXT UMFLOSSEN
\begin{wrapfigure}[10]{r}{8cm}
\centering
\includegraphics[width=7cm]{media/bragg}
\caption{\textsl{Aufbau eines Bragg-Spektrometers}}
\label{fig:bragg}
\end{wrapfigure}

%TABELLEN
\begin{table}[h]
\centering
\begin{tabular}{|c|c|}
\hline
$z$ [mm] & $\Delta z$ [mm]\\
\hline
3,3 & 0 \\
4,9 & 1,6 \\
6,5 & 1,6 \\
8,0 & 1,5 \\
\hline
\end{tabular}
\caption{Position der Minima zueinander und jeweiliger gemessener Abstand}
\label{tab:label}
\end{table}

% VERBUNDENE ZELLEN IN TABELLEN
\begin{table}[h]
\centering
\begin{tabular}{|c|c|c|c|p{1cm}p{1cm}p{1cm}p{1cm}p{1cm}p{1cm}p{1cm}|}
\hline
A & B & C & D & \multicolumn{7}{|c|}{F}  \\ \hline
\multirow{ 2}{*}{1} & 0 & 6 & 230 & 35 & 40 & 55 & 25 & 40 & 35 & \\
& 1 & 5 & 195 & 25 & 50 & 35 & 40 & 45 &  &  \\ \hline
\end{tabular}
\caption{A test caption}
\label{tab:table2}
\end{table}

% MEHRERE TABELLEN NEBENEINANDER
\begin{table}[h]
\centering

\subfloat[PVC, $d=15,0\,$mm]{
\begin{tabular}[b]{|c|c|c|}
\hline
mit Platte & ohne Platte & $\Delta s$\\
\hline
3,6 & 4,85 & 1,25\\
3,55 & 4,8 & 1,25\\
3,5 & 4,75 & 1,25\\
3,55 & 4,85 & 1,3\\
\hline
\end{tabular}
}

\subfloat[PE, $d=12,2\,$mm]{
\begin{tabular}[b]{|c|c|c|}
\hline
mit Platte & ohne Platte & $\Delta s$\\
\hline
3,5 & 4,2 & 0,7\\
3,5 & 4,25 & 0,75\\
3,5 & 4,2 & 0,7\\
\hline
\end{tabular}
}

\caption{Materialproben und zugehörige Messwerte in [cm]}
\end{table}

% FLOATS ERZWINGEN = TABELLEN UND BILDER ZWINGEND EINFÜGEN
\FloatBarrier

% ABSÄTZE MIT EINSTELLBARER UND NACHTRÄGLICH GLOBAL ÄNDERBARER DISTANZ
\mypar

% PDFs ANHÄNGEN
\includepdf[pages=-]{media/protokoll}

%BILDER UND TABELLEN REFERENZIEREN
\ref{labelname}
\pageref{labelname}

%REFERENZIERUNGSREGELN ZUR ÜBERSICHTLICHKEIT
bilder: fig:label
tabellen: tab:label
gleichungen: eq:label

\end{comment}
%%%%%%%%%%%%%%%%%%%%%%%%%%%%%%%%%%%%%%%%%%%%%%%%%%%%%%%%%%%%%%%%%%%%%%%%%%%%%%%%%%%%%%%%%%%%%%%%%%%%%%%%%%%
%%%%%%%%%%%%%%%%%%%%%COMMENT%%%%%%%%%%%%%%%%%%%%%%COMMENT%%%%%%%%%%%%%%%%%%%%%%COMMENT%%%%%%%%%%%%%%%%%%%%%
%%%%%%%%%%%%%%%%%%%%%%%%%%%%%%%%%%%%%%%%%%%%%%%%%%%%%%%%%%%%%%%%%%%%%%%%%%%%%%%%%%%%%%%%%%%%%%%%%%%%%%%%%%%
